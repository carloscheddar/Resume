%%%%%%%%%%%%%%%%%%%%%%%%%%%%%%%%%%%%%%%%%
% Friggeri Resume/CV
% XeLaTeX Template
% Version 1.0 (5/5/13)
%
% This template has been downloaded from:
% http://www.LaTeXTemplates.com
%
% Original author:
% Adrien Friggeri (adrien@friggeri.net)
% https://github.com/afriggeri/CV
%
% License:
% CC BY-NC-SA 3.0 (http://creativecommons.org/licenses/by-nc-sa/3.0/)
%
% Important notes:
% This template needs to be compiled with XeLaTeX and the bibliography, if used,
% needs to be compiled with biber rather than bibtex.
%
%%%%%%%%%%%%%%%%%%%%%%%%%%%%%%%%%%%%%%%%%

\documentclass[print]{friggeri-cv} % Add 'print' as an option into the square bracket to remove colors from this template for printing

% \addbibresource{bibliography.bib} % Specify the bibliography file to include publications

\begin{document}

\header{Carlos}{FelicianoBarba}{Software Developer} % Your name and current job title/field

%----------------------------------------------------------------------------------------
%	SIDEBAR SECTION
%----------------------------------------------------------------------------------------

\begin{aside} % In the aside, each new line forces a line break
\section{Address}\small
Calle Holy Cross 374
Reparto Universitario
San Juan, PR 00926
\section{Contact}
\href{mailto:c.feliciano2009@gmail.com}{\footnotesize c.feliciano2009@gmail.com}
\href{https://github.com/carloscheddar}{\footnotesize github.com/carloscheddar}
\footnotesize cel: +1 (787) 662-4474
\section{Languages}
$\bullet$ English \
$\bullet$ Spanish
% \section{programming}
% {\color{red} $\varheartsuit$} JavaScript
% Python, C++, PHP
% CSS3 \& HTML5
\end{aside}

%----------------------------------------------------------------------------------------
%	Profile
%----------------------------------------------------------------------------------------

\section{Profile}

%------------------------------------------------
\begin{itemize}\small
\item Experience with Backend Web Development including Node.js, Ruby on Rails and Django
\item Proficient use of Git, Github, Heroku, Openshift, Azure and other Cloud
Platforms
\item Proficient programmer in Python, C++, JavaScript, Ruby, SQL, Scheme
\item Experience installing hardware and software
\item Skilled with UNIX based systems \\
\end{itemize}
%------------------------------------------------

%----------------------------------------------------------------------------------------
%	Experience
%----------------------------------------------------------------------------------------

\section{Experience}

\begin{entrylist}
%------------------------------------------------
\entry
{\footnotesize Winter 2013}
{UPR Grade Entry System at University of Puerto Rico}
{San Juan, Puerto Rico}
{\small \emph{Backend Web Developer and System Administrator} \\
\small Co-Created the Web based Grade Entry System for the Río Piedras campus, prior to this it
it was done by hand. \\
\small {\bf Source:} \href{https://github.com/crzrcn/UPRRP-Grades-Entry-System}{github.com/crzrcn/UPRRP-Grades-Entry-System}}
%------------------------------------------------
\entry
{\footnotesize October 2013}
{Ride Surfing at Startup Weekend}
{San Juan, Puerto Rico}
{\small \emph{1\textsuperscript{st} Place} \\
\small My team participated and won the San Juan Startup Weekend by making Ride Surfing, an open source
carpooling platform. \\
\small {\bf Demo:} \href{http://ridesurfing.herokuapp.com/}{ridesurfing.herokuapp.com}}
%------------------------------------------------
\entry
{\footnotesize Summer 2013}
{Descartes-BI at PR Office of Budget and Management}
{San Juan, Puerto Rico}
{\small \emph{Backend Web Developer Intern} \\
\small Connected the LIBRE API to Descartes-BI by adding functionality of RESTful queries that were used to extract and graph government data.\\
\small {\bf Source:} \href{https://github.com/rosarior/descartes-bi}{github.com/rosarior/descartes-bi}}
%------------------------------------------------
\entry
{\footnotesize May-June 2013}
{Databases project at University of Puerto Rico}
{San Juan, Puerto Rico}
{\small \emph{Independent Web Developer} \\
\small Created a website where we modified open source games to dynamically save
data in a PostgreSQL database using Ruby on Rails and deployed on Heroku. \\
\small {\bf Source:} \href{https://github.com/carloscheddar/WebGame}{github.com/carloscheddar/WebGame}}
%------------------------------------------------
\entry
{\footnotesize Jan 2012-2013}
{Computer Research Lab at University of Puerto Rico}
{San Juan, Puerto Rico}
{\small \emph{Software Developer} \\
\small Parallelized algorithms using Cilk to increase preformance of a linear system solver.}\\
%------------------------------------------------
\end{entrylist}

%----------------------------------------------------------------------------------------
%	Technical Experience
%----------------------------------------------------------------------------------------

\section{Technical Experience}
\begin{itemize}\small
\item Developed the backend of several websites using Sails.js, Express, Ruby on Rails and Django
\item Developed a Distributed File System for a class project in Python and MySQL
\item Open-Sourced scripts \& tools mostly in Node, Python, Bash, C++ and Ruby\\
\end{itemize}


%----------------------------------------------------------------------------------------
%	Languages, Frameworks & Tools SECTION
%----------------------------------------------------------------------------------------

\section{Languages, Frameworks \& Tools}

\setlength{\tabcolsep}{20pt}

\small
\begin{tabular}{lllll}
    Python   & Ruby           & Git    & jQuery   & Sails.js \\
    Django   & C/C++  & Node.js        & HTML/CSS    & Angular.js   \\
    mongoDB  & Ruby on Rails          & JavaScript & CoffeeScript & PostgreSQL
\end{tabular}

% % ---------------------------------------------------------------------------------------
% % Recent Activities SECTION
% %----------------------------------------------------------------------------------------

% \section{recent activities}

% \begin{itemize}\small
% \item Attended the Puerto Rico TechSummit where I created Receptionist, a web app that lets businesses display their schedules and users can make bookings based on the business calendar. (June 2013)
% \item Took part of HackPR, Mayagüez, Puerto Rico's Second Official Hackathon (March 2013)
% \end{itemize}

\end{document}